\documentclass[12pt]{article}

% Text Formating
\usepackage[utf8]{inputenc}
\usepackage[margin=1in]{geometry}
\usepackage[titletoc,title]{appendix}

% Math
\usepackage{amsmath,amsfonts,amssymb,mathtools}

% Graphs and Images
\usepackage{graphicx,float}

% Algorithms
\usepackage[ruled,vlined]{algorithm2e}
\usepackage{algorithmic}

\title{ALGC - Linguagem}
\author{André Vaz}

\begin{document}

    \maketitle

    \section{Especificações do Programa}
        \textbf{Pré-condição}
            É uma proposição lógica que se assume verdadeira
            no início da execução do programa.\\
        \textbf{Pós-condição}
            É uma proposição lógica que se assume verdadeira
            no fim da execução do programa.\\
        \textbf{Triplos de Hoare}
            Um triplo de Hoare escreve-se como $\{P\}\ C\ \{Q\}$, onde
            $P$ é a \textit{pré-condição}, $C$ é o conjunto de \textit{comandos} a executar
            e $Q$ é a \textit{pós-condição}
            \begin{center}\textbf{Exemplo (Triplo de Hoare)}\\ \fbox{
                $\{j=j_0\}\ j=j+1\ \{j=j_0+1\}$
            }\end{center}

    \section{Definir Linguagem de Programação (Imperativa)}
        Para podermos escrever os algoritmos é necessário definir
        uma linguagem. Neste caso iremos usar uma linguagem do tipo
        imperativa.
        \subsection{Atribuição}
            $$x = E$$
            \begin{enumerate}
                \item Calcular o valor de E.
                \item Colocar o resultado desse cálculo em x.
            \end{enumerate}
            \begin{center}\textbf{Exemplo (Atribuição)}\\ \fbox{
                $x = x + 4$
            }\end{center}

\end{document}